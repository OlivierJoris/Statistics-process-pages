\documentclass[a4paper, 11pt, oneside]{article}

\usepackage[utf8]{inputenc}
\usepackage[T1]{fontenc}
\usepackage[english]{babel}
\usepackage{array}
\usepackage{shortvrb}
\usepackage{listings}
\usepackage[fleqn]{amsmath}
\usepackage{amsfonts}
\usepackage{fullpage}
\usepackage{enumerate}
\usepackage{enumitem}
\usepackage{graphicx}
\usepackage{subfigure}
\usepackage{alltt}
\usepackage{indentfirst}
\usepackage{eurosym}
\usepackage{listings}
\usepackage{titlesec, blindtext, color}
\usepackage{float}
\usepackage[colorlinks, linkcolor=blue]{hyperref}
\usepackage[nameinlink,noabbrev]{cleveref}

\usepackage{titling}
\renewcommand\maketitlehooka{\null\mbox{}\vfill}
\renewcommand\maketitlehookd{\vfill\null}

\definecolor{mygray}{rgb}{0.5,0.5,0.5}
\definecolor{pink1}{rgb}{0.858, 0.188, 0.478}
\definecolor{sienna}{rgb}{0.53, 0.18, 0.09}
\definecolor{sepia}{rgb}{0.44, 0.26, 0.08}
\definecolor{midnightblue}{rgb}{0.1, 0.1, 0.44}

\renewcommand{\lstlistingname}{Code}

\lstset{
    language=C, % Utilisation du langage C
    commentstyle={\color{MidnightBlue}}, % Couleur des commentaires
    frame=single, % Entoure le code d'un joli cadre
    rulecolor=\color{black}, % Couleur de la ligne qui forme le cadre
    numbers=left, % Ajoute une numérotation des lignes à gauche
    numbersep=5pt, % Distance entre les numérots de lignes et le code
    numberstyle=\tiny\color{mygray}, % Couleur des numéros de lignes
    basicstyle=\tt\footnotesize, 
    tabsize=3, % Largeur des tabulations par défaut
    extendedchars=true, 
    captionpos=b, % sets the caption-position to bottom
    texcl=true, % Commentaires sur une ligne interprétés en Latex
    showstringspaces=false, % Ne montre pas les espace dans les chaines de caractères
    escapeinside={(>}{<)}, % Permet de mettre du latex entre des <( et )>.
    inputencoding=utf8,
    literate=
  {á}{{\'a}}1 {é}{{\'e}}1 {í}{{\'i}}1 {ó}{{\'o}}1 {ú}{{\'u}}1
  {Á}{{\'A}}1 {É}{{\'E}}1 {Í}{{\'I}}1 {Ó}{{\'O}}1 {Ú}{{\'U}}1
  {à}{{\`a}}1 {è}{{\`e}}1 {ì}{{\`i}}1 {ò}{{\`o}}1 {ù}{{\`u}}1
  {À}{{\`A}}1 {È}{{\`E}}1 {Ì}{{\`I}}1 {Ò}{{\`O}}1 {Ù}{{\`U}}1
  {ä}{{\"a}}1 {ë}{{\"e}}1 {ï}{{\"i}}1 {ö}{{\"o}}1 {ü}{{\"u}}1
  {Ä}{{\"A}}1 {Ë}{{\"E}}1 {Ï}{{\"I}}1 {Ö}{{\"O}}1 {Ü}{{\"U}}1
  {â}{{\^a}}1 {ê}{{\^e}}1 {î}{{\^i}}1 {ô}{{\^o}}1 {û}{{\^u}}1
  {Â}{{\^A}}1 {Ê}{{\^E}}1 {Î}{{\^I}}1 {Ô}{{\^O}}1 {Û}{{\^U}}1
  {œ}{{\oe}}1 {Œ}{{\OE}}1 {æ}{{\ae}}1 {Æ}{{\AE}}1 {ß}{{\ss}}1
  {ű}{{\H{u}}}1 {Ű}{{\H{U}}}1 {ő}{{\H{o}}}1 {Ő}{{\H{O}}}1
  {ç}{{\c c}}1 {Ç}{{\c C}}1 {ø}{{\o}}1 {å}{{\r a}}1 {Å}{{\r A}}1
  {€}{{\euro}}1 {£}{{\pounds}}1 {«}{{\guillemotleft}}1
  {»}{{\guillemotright}}1 {ñ}{{\~n}}1 {Ñ}{{\~N}}1 {¿}{{?`}}1
}


\newcommand{\ClassName}{INFO-0940: Operating Systems}
\newcommand{\ProjectName}{Project 2: Adding system calls - Report}
\newcommand{\AcademicYear}{2020 - 2021}

%%%% Page de garde %%%%

\title{\ClassName\\\vspace*{0.8cm}\ProjectName\vspace{0.8cm}}
\author{Goffart Maxime \\180521 \and Joris Olivier \\ 182113}
\date{\vspace{1cm}Academic year \AcademicYear}

\begin{document}

%%% Page de garde %%%
\begin{titlingpage}
{\let\newpage\relax\maketitle}
\end{titlingpage}

%%%%%%%%%%%%%%%%%%%%%%%%%%%%%%%%%%%%%%%%%%%%%%

\section{Implementation}

\section{How the memory of a process is managed by the kernel ?}

\paragraph{}Processes are implemented in the kernel as instances of \texttt{task\_struct}. 
This struct contains a \textit{mm} field which is an instance of \texttt{mm\_struct} that represents a 
summary of the process memory. This \textit{mm} field contains a \textit{mmap} that is an instance of \texttt{vm\_area\_struct} that represents 
a memory area. This \textit{nmap} field contains 3 fields \textit{vm\_start}, \textit{vm\_end}, and \textit{vm\_next}. The first one represents the logical 
address corresponding to the first address within the virtual memory area, the second one represents the first address outside the virtual memory area, and the 
third one is a reference to another \texttt{vm\_area\_struct} which contains higher segment begin and end addresses. 

\paragraph{}To translate a logical address to a physical address, a page table is used. This table allows 
to obtain the physical address corresponding to a given logical given address using multi-level paging in order to reduce the size of the table that is stored in physical memory.
Moreover, the offset of the logical address is retained according to the page size and only the first bits of the logical address are used to look at the mapping in the table in order to 
reduce the size of the entries in the table. Some entries do not refer to any physical page : this means that they have the present flag clear. This could be because their contents have been swapped out or because they never have been touched. 

\paragraph{}This management of the process memory by the kernel is represented on the \autoref{fig:kernel_memo}.
 
\begin{figure}[H]
  \centering
  \includegraphics[scale=0.6]{kernel_memo.png}
  \caption{Management of the process memory by the kernel.}\label{fig:kernel_memo}
\end{figure}


\end{document}
